In der vergangenen Session haben wir uns mit einigen Grundlagen zum Verhalten idealer Gase auseinandergesetzt. Dem idealen Gas wird die Abwesenheit dreier Eigenschaften unterstellt:
\begin{wrapfigure}{r}{6cm}
%\vspace{5mm}
%\begin{minipage}[t]{0.4\textwidth}
    \centering
    \includegraphics[width=0.4\textwidth]{entries/2/table.png}
    \label{fig:table}
    \captionof{figure}{Begriffskonventionen}
%\end{minipage}
\end{wrapfigure}

\textbf{1.} Die Gasmoleküle werden als dimensionslos \mbox{betrachtet}. \newline 
\textbf{2.} Keine Wechselwirkung zwischen den \mbox{Molekülen}. \newline 
\textbf{3.} Vollständiger Energieübertrag bei Stößen \mbox{Molekül-Molekül} und Molekül-Gefäßwand.\newline

Reale Gase seien \textit{``bei hohen Temperaturen und geringen Dichten in guter Näherung als ideale Gase''} betrachtbar. Welche Eigenschaften \underline{haben} ideale Gase?
Nach \mbox{Edme} \mbox{Mariotte} (1676) und (unabhängig) \mbox{Robert} \mbox{Boyle} (1662) (vgl. \textit{to boil} - Zufall?) gilt bei konstanter Temperatur \(T\) und Teilchenzahl \(N\)

\begin{equation}
    p_1V_1=p_2V_2
\end{equation}

und nach Joseph Louis Gay-Lussac (\textdagger 1850) für konstanten Druck und Teilchenzahl

\begin{equation}
    \frac{V_2}{V_1}=\frac{T_2}{T_1}
    \label{eq:gay}
\end{equation}

Etwas umgeformt und -benannt führt Gl. \ref{eq:gay} zu der Volumenänderung \(\Delta V\) in Abhängigkeit der Temperaturänderung \(\Delta T\) skaliert durch den Quotienten aus
Anfangsvolumen und -temperatur.

\begin{equation}
    \Delta V = V_1 \frac{1}{T_1} \Delta T = V_1 \gamma \Delta T
\end{equation}
\(\gamma\) wird hier als Temperaturausdehnungskoeffizient der idealen Gase bezeichnet und ist auf den Kehrwert des Nullpunktes der Celsius-Skala in Kelvin \(\vartheta  = \SI{0}{\celsius}\) festgelegt.
Das macht durchaus Sinn, denn wäre er über Null Kelvin definiert, führt das einerseits zu einem mathematischen Problem und andererseits wäre die Volumenänderung durch Temperaturänderung nurnoch in
eine Richtung mathematisch beschreibbar. Im Falle realer Gase kommen noch weitere andere Faktoren zum tragen (siehe oben) die dazu führen, dass sich zum einen der Zusammenhang weder proportional, noch linear
verhält.\\

Was haben wir noch gelernt? Es gibt etwas namens universelle Gaskonstante \(R\). Sie ist das Produkt aus der Avogadro-Zahl \(N_A\) (hallo Chemie) und der Boltzmann-Konstante \(k_B\). Leider
ging das Skript nicht weiter auf die Herkunft von \(k_B\) als Naturkonstante ein. Halliday et. al. (2003) hat dazu zu sagen, dass sie über \(k_B = \frac{R}{N_A}\) definiert ist. 
Im Vergleich zum Skript erst mal eine Tautologie. Ein Gegencheck mit der englischsprachigen Wikipedia liefert eine deutlich komplexere Definition:

\be
    S = k_B \ln{W}
\ee

Wobei \(S\) ein Maß für die Entropie eines makroskopischen Systems und \(W\) ein Maß für die statistische Wahrscheinlichkeit eines mikroskopischen Zustands (eines einzelnen Bestandteils des Systems?) ist.
Da völlig durch zu steigen delegiere ich mal an Zukunfts-Dennis.\\
Herr Boltzmann taucht auch nochmal in der Gleichung für die Gesamtenergie eines Gases auf.

\be
    \bar{E}_{ges} = \frac{1}{2}f k_B T = \bar{E}_{kin}
\ee

\(f\) ist hierbei die Anzahl der Freiheitsgrade der Teilchen innerhalb des Gases.

Zum Abschluss zur Aufheiterung noch ein Bild zu CO2 in überkritischem Zustand:

\begin{figure}[h]
    \centering
    \includegraphics[width=0.6\textwidth]{entries/2/critical.png}
    \caption{Gelb: ein Hauch von Phasenübergang}
\end{figure}
Das Video dazu gibts \textcolor{blue}{\hyperlink{https://www.youtube.com/watch?v=RmaJVxafesU}{hier}}.