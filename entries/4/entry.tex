Grüße Logbuch.

Das versprochene Experiment stellt sich als weit weniger trivial heraus, als naiverweise vermutet. Die Problematik
besteht einerseits darin, den richtigen Entnahmezeitpunkt des *geeignete Flüssigkeit oder Lösung eintragen* zu treffen und andererseits
in der simplen Tatsache, dass es mir Gefrierfachwand-bedeckende-Gletscher bedingt unmöglich ist eine der Schubladen
erschütterungsfrei zu öffnen. Um letzteres anzugehen werden ich mich kommende Woche auf TK-Diät setzen um das Gefrierfach
abtauen zu können. Den Zeitpunkt versuche ich danach iterativ und mit mehreren Proben parallel zu ermitteln.\par
Darüber hinaus werde ich mal nach sehen, wie sich die Temperaturänderung des flüssigen Wassers während eines Siedeverzuges
verhält zu untersuchen. Hier bin ich mir noch nicht ganz sich wie ich einen Siedeverzug halbwegs reproduzierbar provozieren
kann aber Kochtöpfe sind aufgrund ihrer zu rauen Oberfläche schonmal raus und Herdplatten oder Gaskocher als Wärmequelle
auch da sie die Flüssigkeit alles andere als homogen erwärmen. Glas, Mikrowellenherd und der Mission angemessene Handschuhe
werden es wohl sein.\par
Vorlesung! Welche Erkenntnisse gab es diese Woche? Über den ersten Hauptsatz der Thermodynamik haben wir ja bereits gesprochen.
Er besagt, dass gilt:

\be
dU = dQ + dW
\ee

Hier sind \(dU\) die Änderung der inneren Energie eines Systems, \(dQ\) die Änderung seiner Wärmemenge und \(dW\) die an oder
mit ihm verrichtete Arbeit. Letztere wird in den allermeisten Fällen bedingt durch die Änderung des Volumens in Gestalt von mechanischer
Arbeit entnommen. Am Beispiel einer Wärmepumpe kann sie aber auch als thermische Arbeit in Form des Transportes von Wärmeenergie genutzt werden.
Hier wird - wenn ich keinen Denkfehler habe und ganz grob - zwischen isobarer und isothermer Zustandsänderung gewechselt (siehe pV Nerd).
Natürlich läuft solch ein Prozess nur in den Köpfen der Schwurbler von alleine ab. Im Falle eines Kühlschranks wird vom Kompressor während der
isochoren Zustandsänderung Arbeit am System verrichtet: \(\frac{p}{T}=\frac{nR}{V}=const\). Der Druck steigt, die Temperatur des Mediums sinkt.
Dort, wo das Medium verdampft findet wiederum eine isobare Zustandsänderung statt: \(\frac{V}{T}=\frac{nR}{p}=const\). Das Volumen sinkt, die
Temperatur des Mediums steigt (und kühlt seine Umgebung).\par
Nicht wirklich neu aber immernoch ein Mysterium ist mir der entrope Teil der isentropen Zustandsänderung. Womöglich kann ich das Wochenende über
zu Weisheit gelangen.\par

\begin{center}
    Stay tuned :)
\end{center}