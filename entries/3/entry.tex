Hallo Logbuch. Es wird dir sicherlich nicht gefallen aber ich kann dir versichern, dass es
nicht das ist, was ich wollte. Vielleicht beruhigt es dich aber, dass ihr beide ziemlich
viel gemein habt. Thematisch wie intellektuell kann ich vieles mit euch beiden teilen
und es wäre schön wenn wir uns - irgendwann - zu dritt begegnen könnten. Lass sehen, ich denke
ich kann dich mit einer Story aufheitern. Diese Woche hatte ich mehrere sehr seltsame und
teils hitzige Diskusionen mit der Geschäftsleitung der Firma in der ich arbeite. Ich arbeite
seit einigen Monaten an drei Projekten parallel. Eines davon beinhaltet Dimensionierung, Bau
und Kalibrierung eines 3D-Druckers. Dieser Drucker soll ein Druckbett mit den Maßen \SI{350x450x8}{mm}
(Länge x Breite x Stärke) erhalten. Da es ein beheiztes Druckbett ist wird eine \SI{650}{W}
Silikonheizmatte an die Unterseite geklebt. Seiner Meinung nach (die Geschäftsleitung ist ein Dude)
sei es nicht möglich in endlicher Zeit mit dieser Heizung das Druckbett auf \(T_2 = \SI{80}{\celsius}\) zu bringen.
Tatsache?\par
Die spezielle Legierung mal außer Acht gelassen hat Aluminium laut Wikipedia eine Dichte von
\( \rho = 2,7 \frac{g}{cm^{3}}\). Mit den gegebenen Außenmaßen ergibt das etwa eine Masse von
\(m_{Al} = \SI{3,4}{kg}\). Folie 57 der VL kann ich eine spezifische Wärmekapazität für Aluminium
von \(c_{Al} = \SI{900}{\frac{J}{kg*K}}\) entnehmen. Haben wir damit alles? Ach ja, weiter ziemlich nützlich
ist der Zusammenhang - ebenfalls aus der VL auf Folie 53 - dass
\begin{equation}
    \Delta Q = cm \Delta T
\end{equation}
mit \(\Delta Q\) als zugeführter Wärmemenge, \(c\) als spezifischer Wärmekapazität, \(m\) als Masse und \(\Delta T\) als
Änderung der Temperatur in Kelvin. Ich habe nur überschlagsmäßig gerechnet. Die Temperaturabhängigkeit 
der Wärmekapazität von Aluminium verläuft in dem betrachteten Temperaturbereich ziemlich flach, also nahm ich sie
als Konstant an. Außerdem ging ich von einer Raumtemperatur von \(T_1 = \SI{25}{\celsius}\) aus.\par
Also: Welche Wärmemenge muss ich in den Aluklotz stecken um ihn hoch zu heizen?
\begin{align}
    \Delta Q    &= c_{Al}m_{Al}\Delta T \nonumber \\
                &= c_{Al}m_{Al}(T_2 - T_1) \nonumber \\
                &= \SI{900}{\frac{J}{kg*K} * \SI{3,4}{kg} * \SI{55}{K}} \nonumber \\
                &= \SI{168300}{J} \nonumber
\end{align}

Mit der Leistung der Heizmatte \(P = \SI{650}{W}\) kommen ich dann auf

\begin{align}
    t   &= \frac{\Delta Q}{P} \nonumber\\
        &= \frac{\SI{168300}{Ws}}{\SI{650}{W}} \nonumber \\
        &= \SI{258,92}{s} = \SI{4,3}{min}
\end{align}

\textbf{In your face!} Da geh ich eben einen Kaffee holen und wenn ich zurück bin ist das Ding warm. Klar, bis sich die Temperatur
homogenisiert hat dauerts auch noch einen Moment aber selbst mit dem Malus sieht es mir doch nach einer ziemlich endlichen
Zeitspanne aus. Deckt sich super mit den heute gemessenen Werten von knapp fünf Minuten. Die Differenz
lässt sich aller Wahrscheinlichkeit nach auf Verluste in den Leitungen und Abweichungen durch diverse Vereinfachungen
erklären.\par\medskip
So, ich muss jetzt auch los. Nein! Nicht das andere Logbuch. Wirklich. Es ist Wochenende - ich geh' ein bisschen
extrovertieren. Mit Menschen.