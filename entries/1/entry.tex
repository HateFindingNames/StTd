Zunächst muss ich mir klar werden, wie sich Strömungslehre und Thermodynamik definieren. Meine ungebildete Vorstellung der Strömungslehre war zunächst,
dass sie sich mit dem Verhalten der Bewegung einer Gesamtheit vieler einzelner Teilchen durch einen Räumlich begrenzten Bereich auseinander setzt.
Damit meine ich beispielsweise den Luftstrom durch einen Lüfter, den Partikelstrom durch eine Turbine, das Verhalten von Flüssigkeiten durch Rohre
aber auch Schüttverhalten verschiedener Medien oder Luftströmungen in der Atmosphäre etc.
Laut Audio liegt der Fokus der LV auf dem Themengebiet der Thermodynamik da es im Leben als Ingenieur mehr Kontaktpunkte gäbe.\\

Das erste merkenswerte ist die getroffene Unterscheidung zwischen Zustandsgröße und -variable. Sind Zustandsvariablen hier
als Teilmenge der Zustandsgrößen zu verstehen? Laut Folie 7 bilden die Zustandsvariablen die Gesamtheit derjenigen Größen, die zur "`eindeutigen Beschreibung eines Systems
ausreichen"'. Wikipedia hat hierzu zu sagen, dass
\begin{quote}
    \textit{"}\textit{eine Zustandsgröße (...) den Zustand eines physikalischen Systems beschreibt,
    \textbf{aber im Rahmen der Betrachtung als Variable angesehen wird.}"}
\end{quote}
Was bedeutet das nun? Angenommen mein System besteht aus einem Treibgas in einer Spraydose, die ich ausreichend lange hab herumstehen lassen, so, dass
sie und ihr Inhalt Raumtemperatur angenommen haben. Welche Größen beschreiben nun mein System eindeutig? Temperatur, Volumen des Gefäßes, Druck, Dichte
des Treibgases, Volumen des Treibgases, Masse. Unter der Annahme gleich bleibender Raumtemperatur und ohne, dass ich auf die Sprühkappe drücke
hat das System keinen anlass sich zu ändern. Betätige ich allerdings die Sprühkappe ändern sich einige der Zustandsgrößen - aber nicht alle! Das Volumen
des Gefäßes etwa bleibt gleich wärend die übrigen Größen sich ändern. Ich hoffe ich habe das so richtig verstanden. In diesem Zusammenhang ebenfalls merkenswert
ist die Unterteilung extensiver- und intensiver Größen (wobei mir die Erklärung trivial erschien).\\
Was ist Enthalpie? Ein Maß für die \textit{innere Energie} eines Systems. Ich verstehe darunter das Vermögen eines Systems eine oder mehrere Zustandsgrößen
eines anderen Systems ändern zu können. Bspw. eine definierte Menge Wasser die qua Temperatur in der Lage ist die Temperatur einer anderen Menge Wasser
zu ändern sobald ich die beiden Mengen (Systeme) miteinander in Kontakt bringe.\\

\textbf{0. Hauptsatz der Thermodynamik:}
\begin{quote}
    \textit{"}\textit{Alle Systeme, die sich mit einem gegebenem System im thermischen Gleichgewicht befinden, stehen auch untereinander im thermischen Gleichgewicht."}
\end{quote}
Die Erklärung dazu ist recht eingängig. Doch wie lauten die übrigen Hauptsätze? Impliziert die Null, dass der genannte Hauptsatz nachträglich eingeführt und
den anderen übergeordnet wurde (es macht eher den anschein, als wäre hier einfach Informatikerstyle hochgezählt)?

\begin{enumerate}
    \item Hauptsatz: Die Energie innerhalb eines geschlossenen Systems bleibt erhalten.
    \item Hauptsatz: Ein Perpetuum mobile zweiter Art ist unmöglich. (es gibt noch viele andere Formulierungen aber im Grunde ist hier gemeint, dass es nicht
    möglich ist eine gegebene Menge thermischer Energie vollständig in z.b. mechanische Energie umzuwandeln).
    \item Hauptsatz: Es ist unmöglich ein System auf 0 K abzukühlen.
\end{enumerate}

Temperaturskalen: Celsius - intervallskaliert, Kelvin - ratioskaliert, Fahrenheit - intervallskaliert. Sollte man das nicht mittlerweile boykottieren?
Zumindest ist es gut den Umrechnungsfaktor zu kennen: \( T_F(T_C) = \frac{9}{5} T_C + 32 \). Hier ist einerseits erkennbar, dass der
Zusammenhang zwischen °C und °F linear ist und zum anderen, dass bei \SI{32}{} °F der Gefrierpunkt liegt (\SI{0}{} °C). Wir haben nun Einheiten um Temperaturen
angeben zu können. Messbar sind sie auf vielen verschiedenen Wegen. Bemerkenswert ist aber, dass einige Wege umkehrbar während es andere nicht sind. Das Thermoelement
etwa funktioniert basierend auf dem Phänomen der Thermoelektrizität oder dem Seebeck-Effekt im Speziellen. Durch ein Temperaturunterschied \(\Delta t\) wird eine
Spannung \(U_{th}\) erzeugt. Umgekehrt kann aber auch durch eine angelegte Spannung eine Temperaturdifferenz provoziert werden (Peltier-Effekt). Der Wirkungsgrad
ist zwar sehr gering, die erzeugten Temperaturdifferenzen sind bei relativ kleinen Leistungen (\SI{12}{V} und \SI{5}{}-\SI{10}{A}) aber schon beachtlich.\\

Temperaturdifferenzen verursachen aber immer auch räumliche Ausdehnungen oder Kontraktionen des erwärmten oder gekühlten Materials (bezogen auf eine Referenztemperatur \(T_0\)
und -länge \(L_0\)). Die Änderung der Länge \(L(T)\) lässt sich mittels folgender Gleichung beschreiben:

\be
    L(T) = L_0 (1 + \alpha \Delta T) \label{length}\\
\ee

Diese Gleichung beinhaltet den materialspezifischen \textit{Temperaturausdehnungskoeffizienten} \(\alpha\) in der Einheit \(\SI{}{K}^{-1}\), die ursprüngliche Länge \(L_0\)
und die Temperaturdifferenz \(\Delta T\) die sich durch Subtraktion der ursprünglichen Temperatur \(T_0\) von der Zieltemperatur \(T\) ergibt.
Hieraus lassen sich nun auch Flächen- und Volumenausdehnungen ableiten. Im Zuge der LV sollte aus \gl{length} unter der Annahme, dass es sich um ein quadratisches Material handelt,
die gegebene Gleichung für die temperaturabhängige Flächenausdehnung 

\begin{equation}
A(T) \approx  A_0 (1 + 2 \beta \Delta T) \label{area}
\end{equation}

mit

\begin{equation*}
    \beta = 2 \alpha
\end{equation*}

selbst hergeleitet werden.

\begin{align}
    A(T) &= L^{2}(T) = L_{0}^{2} (1 + \alpha \Delta T)^{2} \nonumber\\
    A(T) &= A_0 (1 + 2 \alpha \Delta T + \alpha^{2} \Delta T^{2}) \label{area_exakt}
\end{align}

\gl{area_exakt} liefert hier die exakte Lösung. Da allerdings die materialspezifischen Temperaturausdehnungskoeffizienten für die meisten gängigen Materialen (gibt es welche, die hier stark abweichen?)
mit \(10^{-6}\) eingehen können Therme mit \(\alpha^{n>1} | n \in \mathbb{N} \) in ausreichend guter Näherung vernachlässigt werden. Wenn weiter \(2 \alpha\) mit \(\beta\) substituiert wird führt das genau zu
\gl{area}. Analog hierzu lässt sich auch die Volumenausdehnung mit \(\gamma = 3\alpha\) berechnen.\\
Da temperaturbedinge Längenausdehnungen immer nach außen zeigen kann man sich diesen Umstand zu Nutze machen um beispielsweise eine Welle in eine Passung pressen zu können. Vor dem Fügen
wird hier die Passung mit geeigneten Mitteln erwärmt bzw. die Welle gekühlt.