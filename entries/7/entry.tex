Diese Woche ging es (mir) in erster Linie um den Begriff der Entropie. Ja, haben wir schonmal drüber diskutiert (Eintrag
\#2, Gl. 2.4). Ich bin mir allerdings nicht sicher, ob ich im Vergleich zu damals nennenswert schlauer wurde. Zunächst aber
etwas Kontext: Es ist möglich Prozesse zyklisch ablaufen zu lassen. Bedeutet, dass, mit beliebig vielen Zwischenschritten,
der Endzustand dem Anfangszustand entspricht. Ein Kreisprozess innerhalb eines geschlossenen Systems wird sich allerdings
nicht selbstständig aufrecht erhalten, da Teilschritte eine Senkung der Entropie vorraussetzen und genau hier hat die Natur
Vetorecht. Auf natürliche Weise kann Entropie nur gleich bleiben oder ansteigen (\(\Delta S \geqq 0\)). Mit dem aus dem Weg
kommen wir zum interessanten Teil - was ist Entropie?
\textit{"Die Entropie beschreibt die Zahl der Mikrozustände durch die der beobachtete Makrozustand des Systems realisiert werden kann."} - uff.
Mikrozustände kann jemand auffassen als Möglichkeiten \(n\) verschiedene Socken auf \(N\) Schubladen zu verteilen.
An die Anzahl der Mikrozustände komme ich in diesem Fall mittels
\begin{equation}
    a = N^n
    \label{eq:1}
\end{equation}
Aus Sicht eines sockenaffinen Besuchers wird jedem Mikrozustand (Möglichkeit der Sockenverteilung) die gleiche Wahrscheinlichkeit
zugeordnet. Die Wahrscheinlichkeit ergibt sich also als Kehrwert der Anzahl der möglichen Mikrozustände. Wie verhält sich
die Sache jetzt wenn der seltsame Besucher minimal modisch interessiert ist seiner Meinung nach die Socken daher nicht zu
unterscheiden sind? In dem Fall kann man die konkrete Kombination der Socken in den Schubladen vernachlässigen, sich
bloß auf die Anzahl der Socken in den Schubladen konzentrieren und die übrigen Möglichkeiten Makrozustände nennen.
Klingt simpel aber die Berechnung dazu wird gleich deutlich komplexer.
\begin{equation}
    b = \binom{N+n-1}{n} = \frac{(N+n-1)!}{n! \cdot (N-1)!}
    \label{eq:2}
\end{equation}
Bei zwei Schubladen und fünf Socken (die Waschmaschine hat eine weg gezaubert) ergibt Gleichung \ref{eq:1} 32 Mikrozustände
und Gleichung \ref{eq:2} sechs Makrozustände. Wie verhalten sich nun aber die Wahrscheinlichkeiten bezogen auf die
Makrozustände? In der Theorie, da die Socken nicht unterscheidbar sind, kann jemand bestimmte Mikrozustände zu Makrozuständen
gruppieren. Die Wahrscheinlich für einen Makrozustand ergäbe sich dann als Summe der Einzelwahrscheinlichkeiten innerhalb
seiner Gruppe. Ich habe es aber noch nicht zustande gebracht das in Mathematik zu gießen. Das ist per se schon ärgerlich,
darüber hinaus nervt es aber auch, da \(W\) aus der eingangs genannte Gleichung aus Eintrag \#2 der Quotient aus der
Wahrscheinlichkeit für einen Makrozustand und der Anzahl der Mikrozustände ist. Das schreit nach mehr Epistemologie...