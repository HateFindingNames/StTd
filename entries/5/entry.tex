Logbuch, Donnerstag der 4.6. Anno insoliti.\absatzmid
Wir diskutieren Wärmeübertragung bzw. Wärmeübergang, wie sie sich in Mathematik gießen lassen und ein bisschen ihre physikalischen Ursachen.
Es werden zunächst dreierlei Wege des Wärmetransports unterschieden:
\begin{enumerate}
    \item Transport durch Impuls (Teilchenstöße).
    \item Transport durch Materie (Konvektion).
    \item Transport durch elektromagnetische Emmision/Absorption.
\end{enumerate}

Die Mathematik zum Wärmeübergang kommt im allgemeinen in folgender Gestalt daher
\begin{equation} \label{eq:temp}
    \frac{\Delta Q}{\Delta t} = \dot{Q} = \alpha_K \cdot A \cdot (\Delta T)
\end{equation}
mit
\begin{align*}
    &\dot{Q} & &\text{Wärmestrom}\\
    &\alpha  & &\text{Proportionalitätskoeffizient}\\
    &A         & &\text{Betrachtete Fläche}\\
    &\Delta T  & &\text{Steilheit des Wärmegradienten}
\end{align*}

Der Proporionalitätskoeffizient \(\alpha_K\) kann je nach Situation verschiedene Namen und Werte annehmen. Im Falle
der \fett{Konvektion} handelt es sich um den Wärmeübergangskoeffizienten, im Falle der \fett{Wärmeleitung} durch
(homogene?) Festkörper ist es der Wärmedurchlasskoeffizient \(\Lambda = \frac{\lambda}{d}\) mit \(\lambda\) als Wärmeleitfähigkeit und \(d\) als
Distanz zwischen den beiden betrachteten Wandflächen. Wandflächen? Danke, dass du fragst. Hier wird von einem,
ausgehend von der Grundfläche \(A\), linear extrudierten Körper ausgegangen. Ich schätze, dass die Beschreibung eines
beliebigen Körpers nicht mehr gar so einfach sein wird. Der \fett{Wärmeübergang} ist sehr ähnlich der Wärmeleitung.
Hier wird der Wärmetransport von einem Material in ein anderes beschrieben. Hier wird \(\alpha\) zum \fett{Wärmeübergangskoeffizienten}.
Noch nicht genug? \fett{Wärmedurchgang}\par\medskip

Nachtrag: es gab neuerdings und wie immer extrem dankenswerterweise ein Video zur Thematik.
Interessanterweise lässt sich Gl. \ref{eq:temp} analog zum ohmsche Gesetzt behandeln. Aus \(U R I\) wird dann etwas sperriger \(\Delta T R_{th} \dot{Q}\)
mit
\begin{align*}
    &R_{th} = \frac{1}{\alpha \cdot A} & &\text{für Wärmeübergang und Konvektion}\\
    &R_{th} = \frac{d}{\lambda \cdot A} & &\text{für Wärmeleitung}\\
    &R_{th} = \frac{1}{k \cdot A} & &\text{für Wärmedurchgang}
\end{align*}